% !TEX encoding = UTF-8 Unicode
\documentclass[uplatex,a4j,twocolumn]{jsarticle}

\usepackage{myreport}
\renewcommand{\thefootnote}{*\arabic{footnote}}

\begin{document}

\title{\「流れ学第二\」試験対策委員会標準表記法}
\author{ossan-arrow}
\date{\today}
\maketitle

\section{文章}
	\subsection{文体}
		文体は常体を用いる.
	\subsection{句読点}
		文書において, 読点は半角カンマ, 句点は半角ピリオドを用いる. 句読点の後には, それぞれ2分アキを挿入する.
	\subsection{ハイフン・ダッシュ・マイナス記号}
		一単語を複数行に分けて書くとき, 英単語中に表れるとき, 電話番号・住所・型番等の数字を区切って表記するときにはハイフンを用いる.
		
		区間・範囲を表すときにはenダッシュを用いる.
		
		emダッシュは文書中では使用しない.
		
		時間の経過を表すときや, 説明・副題などを対で囲んで表すときには倍角ダッシュを用いる. ただし, 表示には\code{okumacro.sty}で定義されたものを使用する事とする.\footnote{ターミナルで\『\code{texdoc\char"20 okumacro}\』と入力すると, マニュアルを参照することができる. 他のパッケージについても同様.}
	\subsection{脚注}
		単語や文に脚注を付ける場合, 単語の場合はその単語の直後, 文の場合はピリオドの直後に脚注記号を付す.
\section{数式}
	\subsection{句読点}
		数式は概ね一つの文として扱う. つまり, 最後には句読点が付き物である. 複数の数式を並べて書くときには, 最後の数式以外にカンマを, 最後の数式にピリオドを打つ. その後に文が続く場合には句読点は必要無い. 文の後に数式が続き, かつその内容が連続している場合には, 文の最後に\emph{全角}コロンを打つ.
	\subsection{数式環境}
		1行の数式には\code{equation}環境を, 複数行の数式には\code{align}環境を使用する. 行内の数式は\code{\$}で囲むものとする.
	\subsection{単位}
		単位の付いた物理量については, \code{numprint}パッケージを使用する事とする.
	\subsection{ベクトル}
		ベクトルを文字で表記するときには, \code{\BS{}bm}コマンドを使用する.
	\subsection{行列}
		行列は\code{\BS{}batrix}コマンドを使用する. これを用いることで, ブラケットで囲まれた行列が表現できる. 逆行列については\code{\BS{}inv}, 転置行列は\code{\BS{}tr}, 行列のランクは\code{\BS{}rank}コマンドを使用する.
		
		行列式は\code{\BS{}vatrix}コマンドを使用する.
	\subsection{テンソル}
	\subsection{添え字}

\balance
\end{document}
